\documentclass{beamer}

\usetheme{boxes}
\usecolortheme{rose}
\useinnertheme{circles}
\useoutertheme{split}
\setbeamertemplate{blocks}[rounded]

\usepackage{graphicx}
\usepackage{tikz}

\setbeamertemplate{navigation symbols}{}%remove navigation symbols

%next set colors - not needed
\setbeamercolor{title}{fg=blue!70!black}
%\setbeamercolor{frametitle}{fg=red!70!black}
%\setbeamercolor{framesubtitle}{fg=green!30!black}
%\setbeamercolor{author}{fg=green!50!black}
%\setbeamercolor{institute}{fg=blue!50!black}

\usepackage[slovene]{babel}
\usepackage[OT2,T1]{fontenc}
\usepackage[utf8]{inputenc}
\usepackage{amsmath,amssymb,amsthm}
\usepackage{colortbl}
\usepackage[all]{xy}
\usepackage{array}
\usepackage{amsfonts}

\newtheorem{izrek}{Izrek}
\newtheorem{trditev}{Trditev}
\newtheorem{definicija}{Definicija}
\newcommand{\odvod}[1]{f^{({#1})}(x)}

%\usepackage{pgfpages}
%\setbeameroption{show notes on second screen=right}
\setbeamercovered{invisible}

\title{Števili $e^2$ in $\pi^2$ sta iracionalni}
\author{Ema Češek}
\institute{Seminar, FMF}
\date{29. 3. 2019}

\begin{document}
\frame{\titlepage}

\begin{frame}

\begin{definicija}
 Realna števila, ki niso racionalna, t.j. $\mathbb{R}\backslash\mathbb{Q}$, imenujemo iracionalna števila.
\end{definicija}

\pause
\vskip 1cm

\begin{exampleblock}{Primeri}
\begin{center}
$\sqrt{2}$, $\sqrt{3}$, $\sqrt{5}$,... \\
\vskip 0.5cm
\pause
$\sqrt{2} + \sqrt{3}$, $\sqrt{5} + \sqrt{3}$, $\sqrt{15}$, $\sqrt{6}$,...
\vskip 0.5cm
\pause
$\sqrt{2}\pm 5$, $\frac{1}{\sqrt{3}}$, $-\frac{\sqrt{5}+\sqrt{3}}{7}$,...
\end{center}
\end{exampleblock}
\end{frame}

\begin{frame}
\begin{definicija}
Število $\alpha$ je algebraično, če obstaja neničeln polinom $f(x)=c_nx^n+\dots+c_1x+c$, kjer $c_i\in\mathbb{Z}$ in $f(\alpha)=0$. Če tak polinom ne obstaja, rečemo, da je število transcendentno.
\end{definicija}
\vskip1cm
\pause

\begin{exampleblock}{Primeri}
\begin{center}
$\sqrt{7}$, $\sqrt[3]{5}$, $\sqrt[5]{91}$,...
\vskip 0.5cm
$e$, $\pi$,...
\vskip 0.5cm
\end{center}
\end{exampleblock}
\end{frame}

\begin{frame}
\frametitle{Število $e$}

$e := \displaystyle \lim_{x\to \infty}\bigg(1+\frac{1}{x}\bigg)^x $ \\
\vskip 0.5cm
\quad $\displaystyle = \sum_{k=0}^{\infty}\frac{1}{k!}$
\vskip 0.5cm
\quad $ \displaystyle =2+\frac{1}{1+\frac{1}{2+\frac{1}{1+\frac{1}{1+\frac{1}{4+\frac{1}{1+\frac{1}{1+\frac{1}{6+...}}}}}}}}$ 
\vskip 0.5cm
\[ 2.71828182845904523536028747135266249775724709369995... \]

\end{frame}
 
\begin{frame}
\begin{izrek}
Število $e^2$ je iracionalno.  
\end{izrek}

\vskip 1cm
\pause

\begin{izrek}
Število $e^r$ je iracionalno za vsak $r\in\mathbb{Q}\backslash\{0\}$ .
\end{izrek}
\end{frame}

\begin{frame}
\frametitle{Število $\pi$}

\[ \pi = 3.14159265358979323846264338327950288419716939937510... \]
\vskip 1cm
Približki $\frac{22}{7}$, $\frac{355}{113}$ in $\frac{104348}{33215}$.

\end{frame}

\begin{frame}
\begin{exampleblock}{Primeri}
\begin{center}
 $cos(r)$, $sin(r)$, $tan(r)$,...; $r \in \mathbb{Q}\backslash\{0\}$
\vskip 0.5cm
 $ln(r)$; $r>0$ in $r\ne 1$
\end{center}
\end{exampleblock}
\end{frame}

\setbeamercovered{transparent}
\begin{frame}
\frametitle{}

\begin{izrek}
Število $\pi^2$ je iracionalno.  
\end{izrek}

\pause
\begin{trditev}
Naj bo $n\ge1$ in $f(x)=\frac{x^n(1-x)^n}{n!}$.

\begin{enumerate}
\item Funkcija je oblike $\displaystyle f(x)=\frac{1}{n!}\sum_{i=n}^{2n}c_ix^i$, kjer $c_i\in\mathbb{Z}$.
\pause
\item Za $0<x<1$ velja $0<f(x)<\frac{1}{n!}$.
\pause
\item Vrednosti odvoda $\odvod{k}$ za $x=0$ in $x=1$ sta celi števili za $\forall k\ge0$.
\end{enumerate}

\end{trditev}

\end{frame}
\setbeamercovered{invisible}

\begin{frame}
\begin{exampleblock}{Odprta vprašanja}
\begin{center}
$e+\pi$, $\frac{e}{\pi}$, $2^e$, $\pi^e$, $\pi^{\sqrt{2}}$, $\displaystyle \gamma = \lim_{n\to\infty} \big( \sum_{k=1}^{\infty}\frac{1}{k}-ln(n)\big)$,...
\end{center}
\end{exampleblock}
\end{frame}



\end{document}
