\documentclass[a4paper]{article}
\usepackage[utf8]{inputenc}
\usepackage[slovene]{babel}
\usepackage{lmodern}
\usepackage[T1]{fontenc}
\usepackage{eurosym}
\usepackage{units}
\usepackage{array}
\usepackage{graphicx}
\usepackage{amsfonts}
\usepackage{url}

\usepackage{amsmath} %math
\usepackage{amssymb} %simboli
\usepackage{amsthm} %theorem
\usepackage{indentfirst} %prvi odstavek

\theoremstyle{plain}
\newtheorem{trditev}{Trditev}
\newtheorem{izrek}{Izrek}
\newtheorem{definicija}{Definicija}

\newcommand{\odvod}[1]{f^{({#1})}(x)}
\newcommand{\pina}[1]{\pi^{#1}}

\makeatletter
\newcommand{\mathleft}{\@fleqntrue\@mathmargin0pt}
\newcommand{\mathcenter}{\@fleqnfalse}
\makeatother


\begin{document}
\title{\textbf{ŠTEVILI $e^2$ IN $\pi^2$ STA IRACIONALNI}}
\author{Ema Češek}
\date{}
\maketitle
%\thispagestyle{empty} ni oštevilčeno

\section*{Iracionalna števila}
Množico realnih števil sestavljajo racionalna in iracionalna števila. Racionalna števila so vsa števila, ki jih lahko zapišemo z ulomkom oblike $\frac{a}{b}$, kjer $a\in \mathbb{Z}$ in $b\in \mathbb{Z}\backslash\{0\}$.
\begin{definicija}
Realna števila, ki niso racionalna, t.j. $\mathbb{R}\backslash\mathbb{Q}$, imenujemo iracionalna števila.
\end{definicija}
Nekaj primerov iracionalnih števil:\ $\sqrt{2}$, $\sqrt{3}$, $\sqrt{6}$, $\sqrt{2}+\sqrt{3}$,  $log_23$, $ln2$, $e$, $\pi$,... Več jih bomo spoznali skozi celoten članek. Njihov decimalni zapis se od racionalnih števil razlikuje v tem, da je neskončen in ni periodičen. Manj znano je definiranje iracionalnih števil s pomočjo razdalj do racionalnih števil. Vsako iracionalno število ima do racionalnih same različne razdalje, medtem ko ima vsako racionalno število $r$ enaki že razdalji do $r+1$ in $r-1$. Racionalnih števil je števno neskončno, množica iracionalnih števil pa ima moč kontinuuma. Je gosta, kar pomeni, da med poljubnima dvema realnima številoma vedno najdemo iracionalno število.

%Ni še dokazano, da $2^e$, $\pi^e$, $\pi^{\sqrt{2}}$,...

Odkritje obstoja iracionalnih števil pripisujejo Pitagorejcem, torej sega v čas 5.\ stoletja pr.\ n.\ št. Poznali so jih kot 'neizmerljiva' števila (angleško incommensurable). To pomeni, da ne obstaja enota, s katero bi lahko predstavili dve dolžini kot večkratnik le te. Takšen primer bi bili stranica in diagonala enotskega kvadrata. Dolžina diagonale je namreč $\sqrt{2}$. Geometrijski dokaz 'neizmerljivosti' $\sqrt{2}$ je opisan že v Evklidovih Elementih \cite{knjiznica}. Oznaka za koren se pojavi šele v 16.\ stoletju. 

\begin{izrek}
Število $\sqrt{2}$ je iracionalno.
\end{izrek}
\begin{proof}
Predpostavimo, da je racionalno. Zapišimo ga z okrajšanim ulomkom $\sqrt{2} = \frac{a}{b}$, katerega imenovalec je neničeln. Enakost kvadriramo in pomnožimo z $b^2$:
\begin{equation}
\label{eq:koren2}
2b^2 = a^2.
\end{equation}
Ker je leva stran enačbe deljiva z 2, mora biti tudi desna in to je mogoče le, če je $a$ sodo število. Torej $a=2k$, za poljuben $k$. To vstavimo v enačbo \eqref{eq:koren2} in dobimo
\begin{equation*}
b^2= 2k^2,
\end{equation*}
iz česar sledi, da je tudi $b$ sodo število. To pa je v nasprotju s predpostavko, ker ulomek ni okrajšan.
\end{proof}
Dokaz je bil povzet iz \cite{knjiznica}.

Množica iracionalnih števil ni zaprta za osnovne računske operacije seštevanja, odštevanja, množenja in deljenja. Za primer si oglejmo množenje:\ vemo, da je $\sqrt{2}$ iracionalno število, vendar je produkt $\sqrt{2}$ s samim seboj racionalno število. Iz vsakega iracionalnega števila lahko tvorimo novo iracionalno število tako, da ga pomnožimo, mu prištejemo oziroma odštejemo neničelno racionalno število ali ga korenimo. O tem se lahko prepričamo, če zapišemo števila $xr$, $x+r$ in $\sqrt{x}$, kjer $x$ iracionalno in $r$ neničelno racionalno, v obliki ulomka kot racionalno število. Pri vsakem izrazimo $x$ in tako dobimo, da je iracionalno število enako racionalnemu, kar pa seveda ne drži. Če znova vzamemo za primer število $\sqrt{2}$, lahko na ta način tvorimo iracionalna števila $\sqrt{2}\pm 5$, $\frac{1}{\sqrt{2}}$, $-\frac{\sqrt{2}}{7}$,... Splošneje velja, da so števila oblike $\sqrt[n]{a}$, za $a,n\in\mathbb{N}$, ali naravna ali iracionalna \cite{spletnaknjiga}. 

Iracionalna števila lahko aproksimiramo z racionalnimi s pomočjo verižnih ulomkov. Naj bo $x$ iracionalno število, $\frac{a}{b}$ okrajšan ulomek z $b>0$, ki predstavlja racionalni približek, in velja $|x-\frac{a}{b}|<\frac{1}{2b^2}$. Potem je $\frac{a}{b}$ eden od konvergentov verižnega ulomka za $x$.

Poleg delitve na racionalna in iracionalna števila, poznamo tudi delitev na algebraična in transcendentna števila.
\begin{definicija}
Število $\alpha$ je algebraično, če obstaja neničeln polinom $f(x)=c_nx^n+\dots+c_1x+c$, kjer $c_i\in\mathbb{Z}$ in $f(\alpha)=0$. Če tak polinom ne obstaja, rečemo, da je število transcendentno.
\end{definicija}
Algebraična števila so lahko racionalna ali iracionalna, transcendentna pa so vedno iracionalna. Obstoj transcendentnih je prvič dokazal Liouville leta 1844. Primer algebraičnih iracionalnih števil so števila oblike $\sqrt{a}$, kjer $a$ ni popolni kvadrat, saj so ničle polinoma $x^2-a$. Najbolj znani transcendentni števili sta $e$ in $\pi$.


\section*{Števili $e$ in $e^2$}

Število $e$ definiramo kot 
\begin{equation*}
e:=\lim_{x\to \infty}\bigg(1+\frac{1}{x}\bigg)^x = 2.71828182845904523536028747135266249775724709369995...
\end{equation*}
Predstavimo ga lahko tudi kot neskončno vsoto $ e = \sum_{k=0}^{\infty}\frac{1}{k!}$. Leta 1737 Euler prvi dokaže, da je število $e$ iracionalno \cite{knjiznica}. Njegov dokaz temelji na neskončnih verižnih ulomkih. Število e lahko namreč predstavimo z verižnim ulomkom $[2; 1, 2, 1, 1, 4, 1, 1, 6, 1, 1, 8, 1,\dots, 1, 2n, 1,\dots]$. Dokaz, da je tudi transcendentno, predstavi Charles Hermite leta 1873 \cite{e}. Posebej si bomo v nadaljevanju ogledali število $e^2$. Leta 1840 Liouville objavi dokaz, da je tudi $e^2$ iracionalno število.

Za nekatera števila, predstavljena kot neskončne vsote, dokazov iracionalnosti še ne poznamo. Za primer lahko vzamemo vsoto $\sum_{k=0}^{\infty}\frac{1}{k!+1}$, ki je definirana zelo podobno kot $e$. 

\begin{izrek}
Število $e^2$ je iracionalno.
\end{izrek}

\begin{proof}
Predpostavimo, da je $e^2$ racionalno število. Torej ga lahko zapišemo v obliki $e^2 = \frac{a}{b}$,
kjer je $a \in \mathbb{Z}$ in $b \in \mathbb{N}$. V naslednjem koraku enačbo preoblikujemo v 
\begin{equation*}
be = ae^{-1}
\end{equation*}
in na obeh straneh enačbe pomnožimo z $n!$, kjer je $n$ sodo število in $n \ge 0$. Dobimo
\begin{equation}
n!be = n!ae^{-1}.
\label{eq:osnovna}
\end{equation}
Za funkcijo $e^x$ poznamo razvoj v potenčno vrsto 
\begin{equation*}
 e^x = \sum_{k=0}^{\infty}\frac{x^k}{k!}.
\end{equation*}
V enačbi \eqref{eq:osnovna} lahko števili $e$ in $e^{-1}$ nadomestimo z vrstama
\begin{align*}
e &= 1+\frac{1}{1!}+\frac{1}{2!}+\frac{1}{3!}+\dots \\
e^{-1}&= 1-\frac{1}{1!}+\frac{1}{2!}-\frac{1}{3!}+\dots. 
\end{align*}

Najprej si oglejmo, kaj dobimo na levi strani enačbe:
\mathleft
\begin{equation*}
n!be = n!b \bigg(1+\frac{1}{1!}+\frac{1}{2!}+\frac{1}{3!}+\dots \bigg) 
\end{equation*}
\mathcenter
\begin{multline*}
=n!b \bigg(1+\frac{1}{1!}+\frac{1}{2!}+\dots+\frac{1}{n!}\bigg) \\
+ n!b \bigg( \frac{1}{(n+1)!}+\frac{1}{(n+2)!}+\frac{1}{(n+3)!}+\dots\bigg)
\end{multline*}
\begin{multline*}
=n!b \bigg(1+\frac{1}{1!}+\frac{1}{2!}+\dots+\frac{1}{n!}\bigg)\\
 + b \bigg( \frac{1}{n+1}+\frac{1}{(n+1)(n+2)}+\frac{1}{(n+1)(n+2)(n+3)}+\dots\bigg)
\end{multline*}
Vsoto smo razbili na dva dela. V prvem delu so členi, ki se pojavijo pred $\frac{1}{n!}$. Ker imajo vsi členi imenovalec manjši od $n!$, pri množenju z $n!$ dobimo cela števila. Torej bo prvi del predstavljal neko celo število. Drugi del, kjer smo vzeli vse nadaljne člene, smo že množili z $n!$. Ocenimo:
\begin{align*}
\frac{b}{n+1} &<\frac{b}{(n+1)}+\frac{b}{(n+1)(n+2)}+\frac{b}{(n+1)(n+2)(n+3)}+\dots \\
&<  \frac{b}{(n+1)}+\frac{b}{(n+1)^2}+\frac{b}{(n+1)^3}+\dots\\
&= \frac{b}{n+1} \bigg(1+\frac{1}{n+1}+\frac{1}{(n+1)^2}+ \dots\bigg) \\
&= \frac{\frac{b}{n+1}}{1-\frac{1}{n+1}}\\
&= \frac{b}{n}.
\end{align*}
Uporabili smo formulo za računanje vsote neskončne geometrijske vrste. Dobimo oceno, da je drugi del večji od $\frac{b}{n+1}$ in manjši od $\frac{b}{n}$. Za dovolj velik n, to zagotovo ne bo celo število.

Podobno obravnavamo desno stran enačbe \eqref{eq:osnovna}:
\mathleft
\begin{equation*}
n!ae^{-1} = n!a \bigg(1-\frac{1}{1!}+\frac{1}{2!}-\frac{1}{3!}+\dots \bigg) 
\end{equation*}
\mathcenter
\begin{multline*}
=n!a \bigg(\frac{1}{2!}+\dots+(-1)^n\frac{1}{n!}\bigg)\\
+ n!a (-1)^{n+1} \bigg( \frac{1}{(n+1)!}-\frac{1}{(n+2)!}+\frac{1}{(n+3)!}-\dots\bigg)
\end{multline*}
\begin{multline*}
=n!a \bigg(\frac{1}{2!}+\dots+(-1)^n\frac{1}{n!}\bigg)\\
+ a(-1)^{n+1} \bigg( \frac{1}{n+1}-\frac{1}{(n+1)(n+2)}+\dots\bigg)
\end{multline*}
Prvi del bo znova neko celo število. Spomnimo, da smo na začetku predpostavili, da je $n$ sodo. Torej bo drugi del enak
\begin{equation*}
-a \bigg( \frac{1}{n+1}-\frac{1}{(n+1)(n+2)}+\frac{1}{(n+1)(n+2)(n+3)}-\dots\bigg).
\end{equation*}
Znova želimo ta del oceniti. Na naslednji neenakosti uporabimo formulo za vsoto neskončne geometrijske vrste:
\begin{align*}
& -a \bigg( \frac{1}{n+1}-\frac{1}{(n+1)(n+2)}+\frac{1}{(n+1)(n+2)(n+3)}-\dots\bigg) \\
&< -a \bigg( \frac{1}{n+1}-\frac{1}{(n+1)^2}-\frac{1}{(n+1)^3}-\dots\bigg) \\
&= - \frac{a}{n+1}\bigg(1 -\bigg(\frac{1}{n+1}+\frac{1}{(n+1)^2}+\dots\bigg)\bigg) \\
&= - \frac{a}{n+1}\bigg(1 - \frac{1}{n}\bigg)
\end{align*}
Drugi del je večji od $-\frac{a}{n}$ in za poljubno velik $n$ manjši od 0. Torej bi v enačbi \eqref{eq:osnovna} veljalo, da je za velike $n$ leva stran poljubno malo večja od celega števila, desna stran pa poljubno malo manjša. Vendar enačaj v tem primeru ne velja. Protislovje s predpostavko, da je $e^2$ racionalno število.
\end{proof}

Dokaz je bil povzet iz \cite{dokazi}. Podobno dokaz za $e$ in $e^4$. Splošneje velja naslednji izrek:

\begin{izrek}
Število $e^r$ je iracionalno za vsak neničeln $r\in\mathbb{Q}$ .
\end{izrek}

Izreka ne bomo dokazali. Uporabi se ideje, ki jih bomo navedli, ko bomo dokazovali, da je število $\pina{2}$ iracionalno. 

\section*{Števili $\pi$ in $\pi^2$}
Število $\pi$ je definirano kot razmerje obsega in premera kroga. 
\begin{equation*}
\pi = 3.14159265358979323846264338327950288419716939937510...
\end{equation*}
Znani približki števila $\pi$ so $\frac{22}{7}$, $\frac{355}{113}$ in $\frac{104348}{33215}$.

Leta 1761 je Johann Lambert dokazal, da je $\pi$ iracionalno število. Natančneje, dokazal je, da je za vsak $r$, ki je neničelno racionalno število, $tan(r)$ iracionalno število. Transcendentnost števila $\pi$ dokaže Lindemann leta 1882. Naš dokaz iracionalnosti števila $\pi^2$ bo temeljil na dokazu Ivana Nivena iz leta 1947 in uporabi Nivenovega polinoma. Če dokažemo, da je $\pi^2$ iracionalno sledi, da je $\pi$ iracionalno. Splošneje velja, da je $\pi^n$ iracionalno za vsako naravno število $n$ \cite{iracionalna}. Niven je uspel dokazati iracionalnost $cos(r)$ za vsako neničelno racionalno število $r$. Posledično sta iracionalna tudi $sin(r)$ in $tan(r)$ za enake $r$ ter $ln(r)$ ob pogoju, da $r>0$ in $r\ne 1$. Prav tako so za neničelno racionalno število irracionalne vrednosti inverznih trigonometričnih in hiperboličnih funkcij \cite{knjiznica}. Število $log_ax$ je iracionalno, če sta $a$ in $x$ naravni števili ter ima eno izmed njiju praštevilski faktor, ki ga druga nima.

Za razumevanje dokaza o iracionalnosti števila $\pi^2$ je pomembna naslednja trditev:

\begin{trditev}
\label{tr:funkcijaf}
Naj bo $n\ge1$ in $f(x)=\frac{x^n(1-x)^n}{n!}$.
\begin{enumerate}
\item Funkcija je oblike $\displaystyle f(x)=\frac{1}{n!}\sum_{i=n}^{2n}c_ix^i$, kjer $c_i\in\mathbb{Z}$.
\item Za $0<x<1$ velja $0<f(x)<\frac{1}{n!}$.
\item Vrednosti odvoda $\odvod{k}$ za $x=0$ in $x=1$ sta celi števili za $\forall k\ge0$.
\end{enumerate}
\end{trditev}

\begin{proof}
\begin{enumerate}
\item Po binomskem izreku razvijemo $(1-x)^n$. Dobimo predpis
\begin{equation*}
f(x) = \frac{x^n}{n!}\sum_{k=0}^{n}\binom{n}{k}(-1)^kx^k
\end{equation*}
iz katerega je razvidno, da se $x$ pojavlja v potencah $x^n,\dots, x^{2n}$ in da so koeficienti cela števila.
\item Neenakost $f(x)>0$ očitno velja, saj za $0<x<1$ vrednosti $x^n$ in $(1-x)^n$ pozitivni. Za drugo enakost mora veljati $x^n(1-x)^n<1$. To drži, ker $x\in(0,1)$.
\item Najprej si oglejmo vrednosti $\odvod{k}$. Za $k=0,\dots,n-1$ bo v odvodu še vednov vsakem členu nastopal $x$ s pozitivno potenco, torej bo $f^{(k)}(0)=0$. Za $k=2n+1,\dots$ bo vrednost odvoda prav tako 0, ker odvajamo konstatno funkcijo. Za $k=n,\dots,2n$ bo $f^{(k)}(0) = \frac{k!}{n!}c_k$, ki je celo število. Iz enakosti $f(x)=f(1-x)$ sledi $\odvod{k}= (-1)^k f^{(k)}(1-x)$. Torej $f^{(k)}(1)= (-1)^k f^{(k)}(0)$, ki pa vemo, da je celo število.
\end{enumerate}
\end{proof}

\begin{izrek}
Število $\pina{2}$ je iracionalno.
\end{izrek}
\begin{proof}
Denimo, da je $\pina{2}$ racionalno in ga lahko zapišemo kot $\pina{2}=\frac{a}{b}$, kjer $a,b>0$ celi števili. Definiramo 
\begin{equation*}
F(x):=b^n(\pina{2n}f(x)-\pina{2n-2}\odvod{2}+\pina{2n-4}\odvod{4}-\dots)
\end{equation*}
za funcijo $f(x)$ iz trditve \ref{tr:funkcijaf}. Z izračunom odvodov
\begin{align*}
F'(x)&=b^n(\pina{2n}f'(x)-\pina{2n-2}\odvod{3}+\pina{2n-4}\odvod{5}-\dots) \\
F''(x)&=b^n(\pina{2n}\odvod{2}-\pina{2n-2}\odvod{4}+\pina{2n-4}\odvod{6}-\dots)
\end{align*}
dobimo zvezo
\begin{equation}
F''(x)=-F(x)\pina{2}+b^n\pina{2n+2}f(x).
\label{eq:zveza}
\end{equation}
V $F(x)$ nastopajo členi oblike $b^n\pina{2n-2k}$, ki jih z upoštevanjem osnovne predpostavke preoblikujemo v zapis, iz katerega je razvidno, da so členi cela števila:
\begin{equation*}
 b^n\pina{2(n-k)} = b^n\bigg(\frac{a}{b}\bigg)^{n-k} = a^{n-k}b^k.
\end{equation*}
Po tretji točki trditve \ref{tr:funkcijaf} in zgornji ugotovitvi sledi, da sta vrednosti $F(0)$ in $F(1)$ celi števili.

V naslednjem koraku izračunamo odvod, pri čemer upoštevamo enačbo \eqref{eq:zveza} ter osnovno predpostavko:
\begin{align*}
\frac{d}{dx}&\big( F'(x)sin(\pi x)-\pi F(x)cos(\pi x) \big) \\
&=F''(x)sin(\pi x)+\pi F'(x)cos(\pi x)-\pi F'(x)cos(\pi x)+\pina{2} F(x)sin(\pi x) \\
&=\big( F''(x)+\pina{2} F(x) \big) sin(\pi x) \\
&= b^n\pina{2n+2}f(x)sin(\pi x) \\
&=b^n\bigg(\frac{a}{b}\bigg)^n\pina{2}f(x)sin(\pi x) \\
&=a^n\pina{2}f(x)sin(\pi x).
\end{align*}
Definiramo 
\begin{align*}
N:&=\pi \int_{0}^{1}a^nf(x)sin(\pi x)dx \\
&=\pi \int_{0}^{1}\frac{1}{\pina{2}} \frac{d}{dx}\big( F'(x)sin(\pi x)-\pi F(x)cos(\pi x) \big)dx \\
&=\frac{1}{\pi}\big( F'(x)sin(\pi x)-\pi F(x)cos(\pi x) \big) \big|_0^1 \\
&=\frac{1}{\pi} F'(1)sin(\pi)- F(1)cos(\pi)-\frac{1}{\pi} F'(0)sin(0)+F(0)cos(0)\\
&=F(1)+F(0),
\end{align*}
ki bo tako celo število. 

Ocenimo velikost $N$. Iz druge točke trditve \ref{tr:funkcijaf} vemo, da za $0<x<1$ velja $0<f(x)<\frac{1}{n!}$. Na tem intervalu bo tudi $0<sin(\pi x)<1$. Torej bo $N$ integral pozitivne funkcije z vrednostima 0 v mejah in zato pozitiven. Velja naslednja neenakost
\begin{equation*}
N =  \pi a^n\int_{0}^{1}f(x)sin(\pi x)dx < \pi a^n\frac{1}{n!} <1,
\end{equation*}
če izberemo dovolj velik $n$. Dobilo smo oceno $0<N<1$, hkrati pa vemo, da je $N$ celo število. Protislovje.
\end{proof}
Dokaz je bil povzet iz \cite{dokazi}. Kljub temu, da sta $e$ in $\pi$ iracionalna, pa ostaja vprašanje iracionalnosti odprto še pri številih $e+\pi$, $\frac{e}{\pi}$, $2^e$, $\pi^e$, $\pi^{\sqrt{2}}$,... Enako velja za znano Euler-Mascheronijevo konstanto $\gamma = \lim_{n\to\infty} \big( \sum_{k=1}^{\infty}\frac{1}{k}-ln(n)\big)$ \cite{e}\cite{iracionalna}.

\bibliographystyle{plain}
\bibliography{Literatura}


\end{document}