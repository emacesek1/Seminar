\documentclass[a4paper]{article}
\usepackage[utf8]{inputenc}
\usepackage[slovene]{babel}
\usepackage{lmodern}
\usepackage[T1]{fontenc}
\usepackage{eurosym}
\usepackage{units}
\usepackage{array}
\usepackage{graphicx}
\usepackage{amsfonts}
\usepackage{url}

\usepackage{amsmath} %math
\usepackage{amssymb} %simboli
\usepackage{amsthm} %theorem
\usepackage{indentfirst} %prvi odstavek

\theoremstyle{plain}
\newtheorem{izrek}{Izrek}

\begin{document}
\title{\textbf{Števili $e^2$ in $\pi^2$ sta iracionalni}}
\author{Ema Češek}
\date{}
\maketitle
\thispagestyle{empty} %ni oštevilčeno

\begin{abstract}
Na kratko o mojem članku.
\end{abstract}

\section*{Iracionalna števila}
Na splošno o iracionalnosti števil, zgledi, malo zgodovine. Potem omejim na $e$ in $\pi$.

\section*{Iracionalnost števila $e^2$}
Liouville 1840, podobno za $e^4$

\begin{izrek}
Število $e^2$ je iracionalno.
\end{izrek}

\begin{proof}
Izrek bomo dokazali s protislovjem. Predpostavimo, da je $e^2$ racionalno število. Torej ga lahko zapišemo v obliki
\begin{equation*}
e^2 = \frac{a}{b},
\end{equation*}
kjer je $a \in \mathbb{Z}$ in $b \in \mathbb{N}$. V naslednjem koraku enačbo preoblikujemo v 
\begin{equation*}
be = ae^{-1}
\end{equation*}
in na obeh straneh enačbe pomnožimo z $n!$, kjer je $n$ sodo število in $n \ge 0$. Dobimo

\begin{equation}
n!be = n!ae^{-1}.
\label{eq:osnovna}
\end{equation}

Ker znamo funkcijo $e^x$ razviti v potenčno vrsto $e^x = \sum_{k=0}^{\infty}\frac{x^k}{k!}$, lahko $e$ in $e^{-1}$ zapišemo z vrstama
\begin{align*}
e &= 1+\frac{1}{1!}+\frac{1}{2!}+\frac{1}{3!}+\dots \\
e^{-1}&= 1-\frac{1}{1!}+\frac{1}{2!}-\frac{1}{3!}+\dots 
\end{align*}
in ju vstavimo v enačbo \eqref{eq:osnovna}.

Najprej si oglejmo, kaj dobimo na levi strani enačbe:
\begin{align*}
n!be &= n!b \bigg(1+\frac{1}{1!}+\frac{1}{2!}+\frac{1}{3!}+\dots \bigg) \\
&=n!b \bigg(1+\frac{1}{1!}+\frac{1}{2!}+\dots+\frac{1}{n!}\bigg) + n!b \bigg( \frac{1}{(n+1)!}+\frac{1}{(n+2)!}+\frac{1}{(n+3)!}+\dots\bigg)\\
&=n!b \bigg(1+\frac{1}{1!}+\frac{1}{2!}+\dots+\frac{1}{n!}\bigg) + b \bigg( \frac{1}{(n+1)}+\frac{1}{(n+1)(n+2)}+\frac{1}{(n+1)(n+2)(n+3)}+\dots\bigg)
\end{align*}
Vsoto smo razbili na dva dela. V prvem delu so členi, ki se pojavijo pred $\frac{1}{n!}$. Ker imajo vsi členi imenovalec manjši od $n!$, pri množenju z $n!$ dobimo cela števila. Torej bo prvi del predstavljal neko celo število. Drugi del, kjer smo vzeli vse nadaljne člene, smo že množili z $n!$. Ocenimo
\begin{align*}
\frac{b}{n+1} &< \bigg( \frac{b}{(n+1)}+\frac{b}{(n+1)(n+2)}+\frac{b}{(n+1)(n+2)(n+3)}+\dots\bigg) \\
&< \bigg( \frac{b}{(n+1)}+\frac{b}{(n+1)^2}+\frac{b}{(n+1)^3}+\dots\bigg) \\
&= \frac{b}{n+1} \bigg(1+\frac{1}{n+1}+\frac{1}{(n+1)^2}+ \dots\bigg) \\
&= \frac{\frac{b}{n+1}}{1-\frac{1}{n+1}}\\
&= \frac{b}{n}.
\end{align*}
Uporabili smo formulo za računanje vsote neskončne geometrijske vrste. Dobimo oceno, da je drugi del večji od $\frac{b}{n+1}$ in manjši od $\frac{b}{n}$. Za dovolj velik n, to zagotovo ne bo celo število.

Podobno obravnavamo desno stran enačbe \eqref{eq:osnovna}:
\begin{align*}
n!ae^{-1} &= n!a \bigg(1-\frac{1}{1!}+\frac{1}{2!}-\frac{1}{3!}+\dots \bigg) \\
&=n!a \bigg(\frac{1}{2!}+\dots+(-1)^n\frac{1}{n!}\bigg) + n!a (-1)^{n+1} \bigg( \frac{1}{(n+1)!}-\frac{1}{(n+2)!}+\frac{1}{(n+3)!}-\dots\bigg)\\
&=n!a \bigg(\frac{1}{2!}+\dots+(-1)^n\frac{1}{n!}\bigg) + a(-1)^{n+1} \bigg( \frac{1}{(n+1)}-\frac{1}{(n+1)(n+2)}+\dots\bigg)
\end{align*}
Prvi del bo znova neko celo število. Spomnimo, da smo na začetku predpostavili, da je $n$ sodo.Torej bo drugi del enak
\begin{equation*}
-a \bigg( \frac{1}{(n+1)}-\frac{1}{(n+1)(n+2)}+\frac{1}{(n+1)(n+2)(n+3)}+\dots\bigg).
\end{equation*}
Ocenimo, da je to večje od $-\frac{a}{n}$ in manjše od $-\frac{a}{n+1}$. Torej bi v enačbi \eqref{eq:osnovna} veljalo, da je leva stran malo večja od celega števila, desna stran pa malo manjša. Vendar enačaj v tem primeru ne velja. 
\end{proof}

Dokaz je bil povzet iz \cite{dokazi}. Splošneje velja naslednji izrek.

\begin{izrek}
Število $e^r$ je iracionalno za vsak $r\in\mathbb{Q}$ .
\end{izrek}

Izreka ne bomo dokazali. Dokaže se z uporabo idej, ki jih bomo navedli v dokazu iracionalnosti števila $\pi^2$.

\section*{Iracionalnost števila $\pi^2$}

\begin{izrek}
Število $\pi^2$ je iracionalno.
\end{izrek}
\begin{proof}
Tole bo dokaz. 
\end{proof}
Dokaz je bil povzet iz \cite{dokazi}.

\bibliographystyle{plain}
\bibliography{Literatura}


\end{document}