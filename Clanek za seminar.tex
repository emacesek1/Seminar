\documentclass[a4paper]{article}
\usepackage[utf8]{inputenc}
\usepackage[slovene]{babel}
\usepackage{lmodern}
\usepackage[T1]{fontenc}
\usepackage{eurosym}
\usepackage{units}
\usepackage{array}
\usepackage{graphicx}
\usepackage{amsfonts}
\usepackage{url}

\usepackage{amsmath} %math
\usepackage{amssymb} %simboli
\usepackage{amsthm} %theorem
\usepackage{indentfirst} %prvi odstavek

\theoremstyle{plain}
\newtheorem{trditev}{Trditev}
\newtheorem{izrek}{Izrek}

\newcommand{\odvod}[1]{f^{({#1})}(x)}
\newcommand{\pina}[1]{\pi^{#1}}

\begin{document}
\title{\textbf{Števili $e^2$ in $\pi^2$ sta iracionalni}}
\author{Ema Češek}
\date{}
\maketitle
\thispagestyle{empty} %ni oštevilčeno

\begin{abstract}
Na kratko o mojem članku.
\end{abstract}

\section*{Iracionalna števila}
Na splošno o iracionalnosti števil, zgledi, malo zgodovine. Potem omejim na $e$ in $\pi$.

\section*{Iracionalnost števila $e^2$}
Liouville 1840, podobno za $e^4$

\begin{izrek}
Število $e^2$ je iracionalno.
\end{izrek}

\begin{proof}
Izrek bomo dokazali s protislovjem. Predpostavimo, da je $e^2$ racionalno število. Torej ga lahko zapišemo v obliki
\begin{equation*}
e^2 = \frac{a}{b},
\end{equation*}
kjer je $a \in \mathbb{Z}$ in $b \in \mathbb{N}$. V naslednjem koraku enačbo preoblikujemo v 
\begin{equation*}
be = ae^{-1}
\end{equation*}
in na obeh straneh enačbe pomnožimo z $n!$, kjer je $n$ sodo število in $n \ge 0$. Dobimo

\begin{equation}
n!be = n!ae^{-1}.
\label{eq:osnovna}
\end{equation}

Ker znamo funkcijo $e^x$ razviti v potenčno vrsto $\displaystyle e^x = \sum_{k=0}^{\infty}\frac{x^k}{k!}$, lahko $e$ in $e^{-1}$ zapišemo z vrstama
\begin{align*}
e &= 1+\frac{1}{1!}+\frac{1}{2!}+\frac{1}{3!}+\dots \\
e^{-1}&= 1-\frac{1}{1!}+\frac{1}{2!}-\frac{1}{3!}+\dots 
\end{align*}
in ju vstavimo v enačbo \eqref{eq:osnovna}.

Najprej si oglejmo, kaj dobimo na levi strani enačbe:
\begin{align*}
n!be &= n!b \bigg(1+\frac{1}{1!}+\frac{1}{2!}+\frac{1}{3!}+\dots \bigg) \\
&=n!b \bigg(1+\frac{1}{1!}+\frac{1}{2!}+\dots+\frac{1}{n!}\bigg) + n!b \bigg( \frac{1}{(n+1)!}+\frac{1}{(n+2)!}+\frac{1}{(n+3)!}+\dots\bigg)\\
&=n!b \bigg(1+\frac{1}{1!}+\frac{1}{2!}+\dots+\frac{1}{n!}\bigg) + b \bigg( \frac{1}{(n+1)}+\frac{1}{(n+1)(n+2)}+\frac{1}{(n+1)(n+2)(n+3)}+\dots\bigg)
\end{align*}
Vsoto smo razbili na dva dela. V prvem delu so členi, ki se pojavijo pred $\frac{1}{n!}$. Ker imajo vsi členi imenovalec manjši od $n!$, pri množenju z $n!$ dobimo cela števila. Torej bo prvi del predstavljal neko celo število. Drugi del, kjer smo vzeli vse nadaljne člene, smo že množili z $n!$. Ocenimo
\begin{align*}
\frac{b}{n+1} &< \bigg( \frac{b}{(n+1)}+\frac{b}{(n+1)(n+2)}+\frac{b}{(n+1)(n+2)(n+3)}+\dots\bigg) \\
&< \bigg( \frac{b}{(n+1)}+\frac{b}{(n+1)^2}+\frac{b}{(n+1)^3}+\dots\bigg) \\
&= \frac{b}{n+1} \bigg(1+\frac{1}{n+1}+\frac{1}{(n+1)^2}+ \dots\bigg) \\
&= \frac{\frac{b}{n+1}}{1-\frac{1}{n+1}}\\
&= \frac{b}{n}.
\end{align*}
Uporabili smo formulo za računanje vsote neskončne geometrijske vrste. Dobimo oceno, da je drugi del večji od $\frac{b}{n+1}$ in manjši od $\frac{b}{n}$. Za dovolj velik n, to zagotovo ne bo celo število.

Podobno obravnavamo desno stran enačbe \eqref{eq:osnovna}:
\begin{align*}
n!ae^{-1} &= n!a \bigg(1-\frac{1}{1!}+\frac{1}{2!}-\frac{1}{3!}+\dots \bigg) \\
&=n!a \bigg(\frac{1}{2!}+\dots+(-1)^n\frac{1}{n!}\bigg) + n!a (-1)^{n+1} \bigg( \frac{1}{(n+1)!}-\frac{1}{(n+2)!}+\frac{1}{(n+3)!}-\dots\bigg)\\
&=n!a \bigg(\frac{1}{2!}+\dots+(-1)^n\frac{1}{n!}\bigg) + a(-1)^{n+1} \bigg( \frac{1}{(n+1)}-\frac{1}{(n+1)(n+2)}+\dots\bigg)
\end{align*}
Prvi del bo znova neko celo število. Spomnimo, da smo na začetku predpostavili, da je $n$ sodo. Torej bo drugi del enak
\begin{equation*}
-a \bigg( \frac{1}{(n+1)}-\frac{1}{(n+1)(n+2)}+\frac{1}{(n+1)(n+2)(n+3)}+\dots\bigg).
\end{equation*}
Znova želimo ta del oceniti. Na naslednjih neenakostih uporabimo formulo za vsoto neskončne geometrijske vrste:
\begin{align*}
&-a \bigg( \frac{1}{(n+1)}+\frac{1}{(n+1)^2}+\frac{1}{(n+1)^3}+\dots\bigg) \\
&\quad < -a \bigg( \frac{1}{(n+1)}-\frac{1}{(n+1)(n+2)}+\frac{1}{(n+1)(n+2)(n+3)}-\dots\bigg) \\
&\quad \quad< -a \bigg( \frac{1}{(n+1)}-\frac{1}{(n+1)^2}-\frac{1}{(n+1)^3}-\dots\bigg) \\
&\quad \quad= - \frac{a}{(n+1)}\bigg(1 -\bigg(\frac{1}{(n+1)}+\frac{1}{(n+1)^2}+\dots\bigg)\bigg)
\end{align*}
Ocenimo, da je drugi del večji od $-\frac{a}{n}$ in manjši od $-\frac{a}{n+1}$. Torej bi v enačbi \eqref{eq:osnovna} veljalo, da je leva stran malo večja od celega števila, desna stran pa malo manjša. Vendar enačaj v tem primeru ne velja. 
\end{proof}

Dokaz je bil povzet iz \cite{dokazi}. Splošneje velja naslednji izrek.

\begin{izrek}
Število $e^r$ je iracionalno za vsak $r\in\mathbb{Q}$ .
\end{izrek}

Izreka ne bomo dokazali. Dokaže se z uporabo idej, ki jih bomo navedli v dokazu iracionalnosti števila $\pina{2}$.

\section*{Iracionalnost števila $\pi^2$}

\begin{trditev}
\label{tr:funkcijaf}
Naj bo $n\ge1$ in $f(x)=\frac{x^n(1-x)^n}{n!}$.
\begin{enumerate}
\item Funkcija je oblike $\displaystyle f(x)=\frac{1}{n!}\sum_{i=n}^{2n}c_ix^i$, kjer $c_i\in\mathbb{Z}$
\item Za $0<x<1$ velja $0<f(x)<\frac{1}{n!}$.
\item Vrednosti odvoda $\odvod{k}$ za $x=0$ in $x=1$ sta celi števili za $\forall k\ge0$.
\end{enumerate}
\end{trditev}

\begin{proof}
\begin{enumerate}
\item Po binomskem izreku razvijemo $(1-x)^n$. Dobimo predpis
\begin{equation*}
f(x) = \frac{x^n}{n!}\sum_{k=0}^{n}\binom{n}{k}(-1)^kx^k
\end{equation*}
iz katerega je razvidno, da se $x$ pojavlja v potencah $x^n,\dots, x^{2n}$.
\item
\item Najprej si oglejmo vrednosti $\odvod{k}$. Za $k=0,\dots,n-1$ bo v odvodu še vedno nastopal $x$ s pozitivno potenco, torej bo $f^{(k)}(0)=0$. Za $k=2n+1,\dots$ bo vrednost odvoda prav tako 0, ker ...
\end{enumerate}
\end{proof}

\begin{izrek}
Število $\pina{2}$ je iracionalno.
\end{izrek}
\begin{proof}
Denimo, da je $\pina{2}$ racionalno in ga lahko zapišemo kot $\pina{2}=\frac{a}{b}$, kjer $a,b>0$ celi števili. Definiramo 
\begin{equation*}
F(x):=b^n(\pina{2n}f(x)-\pina{2n-2}\odvod{2}+\pina{2n-4}\odvod{4}-\dots)
\end{equation*}
za funcijo $f(x)$ iz trditve \ref{tr:funkcijaf}. Z izračunom odvodov
\begin{align*}
F'(x)&=b^n(\pina{2n}f'(x)-\pina{2n-2}\odvod{3}+\pina{2n-4}\odvod{5}-\dots) \\
F''(x)&=b^n(\pina{2n}\odvod{2}-\pina{2n-2}\odvod{4}+\pina{2n-4}\odvod{6}-\dots)
\end{align*}
dobimo zvezo
\begin{equation}
F''(x)=-F(x)\pina{2}+b^n\pina{2n+2}f(x).
\label{eq:zveza}
\end{equation}
Po trditvi...

Izračunamo odvod
\begin{align*}
\frac{d}{dx}&\big( F'(x)sin(\pi x)-\pi F(x)cos(\pi x) \big) \\
&=F''(x)sin(\pi x)+\pi F'(x)cos(\pi x)-\pi F'(x)cos(\pi x)-\pina{2} F(x)sin(\pi x) \\
&=\big( F''(x)-\pina{2} F(x) \big) sin(\pi x) ~ *
\end{align*}
in upoštevamo enačbo \eqref{eq:zveza} ter zapis $\pina{2}$ kot racionalnega števila:
\begin{align*}
* &= b^n\pina{2n+2}f(x)sin(\pi x) \\
&=a^n\pina{2}f(x)sin(\pi x).
\end{align*}
Definiramo 
\begin{align*}
N:&=\pi \int_{0}^{1}a^nf(x)sin(\pi x)dx \\
&=\pi \int_{0}^{1}\frac{1}{\pina{2}} \frac{d}{dx}\big( F'(x)sin(\pi x)-\pi F(x)cos(\pi x) \big)dx \\
&=\frac{1}{\pi}\big( F'(x)sin(\pi x)-\pi F(x)cos(\pi x) \big) \big|_0^1 \\
&=\frac{1}{\pi} F'(1)sin(\pi)- F(1)cos(\pi)-\frac{1}{\pi} F'(0)sin(0)+F(0)cos(0)\\
&=F(1)+F(0),
\end{align*}
ki bo tako celo število. Iz trditve vemo, da za $0<x<1$ velja $f(x)>0$. 
\end{proof}
Dokaz je bil povzet iz \cite{dokazi}.

\bibliographystyle{plain}
\bibliography{Literatura}


\end{document}